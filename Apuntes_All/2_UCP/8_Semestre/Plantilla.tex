\documentclass[a4paper]{report} % Formato plantilla

\usepackage{graphicx}
\usepackage[utf8]{inputenc} % Para poder usar caracteres especiales
\usepackage[spanish]{babel} % Para poder usar caracteres especiales

\begin{document} % Inicio del documento Template
  \begin{titlepage}
    \centering
    {\scshape\LARGE Universidad Católica de Pereira\par}
    \vfill
    {\scshape\LARGE PLANTILLA\par}
    \vfill
    {\huge\bfseries Apuntes de clase\par}
    \vfill
    {\Large\itshape NOMBRE\par}
    \vfill
    Docente de la materia\par
	   \textsc{}
    \vfill
    {\Large\today\par}
  \end{titlepage}
%======================================================================
  \tableofcontents % Para crear los índices
    \part{}
      \chapter{}
        \section{}
          \subsection{} 
            \paragraph{}\mbox{} \\
%======================================================================
        \section{}
          \subsection{}
            \paragraph{}\mbox{} \\
              
%----------------------------------------------------------------------
    \part{}
    \part{}
    \part{}
    \part{}
%======================================================================
    \part{Metodología de Calificación}
      \chapter{Cortes}
        \section{Porcentajes evaluativos}
          \paragraph{Cortes 1, 2, 3}\mbox{}\\
            \begin{itemize}
              \item Corte 1
              \begin{enumerate}
                \item Talleres 10\%
                \item Proyecto 10\%
                \item Parcial 15\%
                \item Acumulado 35\%
              \end{enumerate}
              \item Corte 2
              \begin{enumerate}
                 \item Talleres 10\%
                \item Proyecto 10\%
                \item Parcial 15\%
                \item Acumulado 35\%
              \end{enumerate}
              \item Corte 3
              \begin{enumerate}
                \item Talleres 10\%
                \item Proyecto 10\%
                \item Parcial 15\%
                \item Acumulado 35\%
              \end{enumerate}
            \end{itemize}
%======================================================================
\end{document}

