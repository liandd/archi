\documentclass[a4paper]{report} % Formato plantilla

\usepackage{graphicx}
\usepackage[utf8]{inputenc} % Para poder usar caracteres especiales
\usepackage[spanish]{babel} % Para poder usar caracteres especiales

\begin{document} % Inicio del documento Template
  \begin{titlepage}
    \centering
    {\scshape\LARGE Universidad Católica de Pereira\par}
    \vfill
    {\scshape\LARGE Optativa I Procesamiento del Lenguage Natural\par}
    \vfill
    {\huge\bfseries Apuntes de clase\par}
    \vfill
    {\Large\itshape Juan David Garcia Acevedo\par}
    \vfill
    Docente de la materia\par
	   Juan Carlos Blandón \textsc{Andrade}
    \vfill
    {\Large\today\par}
  \end{titlepage}
%======================================================================
  \tableofcontents % Para crear los índices
    \part{Introducción a la Inteligencia Artificial}
      \chapter{Introducción a la asignatura}
        \section{Clase 1 Agosto}
            \paragraph{Procesamiento del lenguaje natural}\mbox{} \\
              La materia esta enfocada a la inteligencia artificial y se presentará un componente muy fuerte de investigación durante el transcurso de la asignatura.
La idea de la asignatura es solucionar problemas y tratar en la medida de lo posible aplicar soluciones que sean de calidad.
              \\Los resultados de aprendizaje de la materia se centrarán en habilidades teórico/practicas. Permitiendo conceptulizar conceptos; entender y poder dar definiciones ya que todo es basado en habilidades que denoten calidad.
              \\El procesamiento del lenguaje como su nombre indica, procesar el lenguaje. El ingeniero debe y tiene que saber gramática, saber escribir, leer y redactar textos.
              \\\\Hoy día se usan transformadores, la idea es poder combinar el sistema de reglas con \textit{Machine Learning} con entrenamiento para hacer seguimientos.(Ya hablaremos de reglas más adelante).\\Usando reglas y patrones. Árboles, semántica y se trabaja bastante.
        \section{Introducción a la IA}
          \paragraph{}\mbox{} \\
            Podemos decir que el tema de la asignatura \textbf{Procesamiento del Lenguaje Natural} arranca con un Paper considerado el \textbf{Génesis en 1943} del PLN, el Paper es \textit{A logical calculus of the ideas immanent in nervious activity}, donde se comienza a hablar sobre el sistema nervioso, se habla sobre el funcionamiento nervioso. Entran autores como \textit{Warren McCulloch, Walter Pitts, Alan Turing} y este señor Alan Turing brinda la programación computacional como Teoría de la Computación.
            \\Esta apareció en 1943 cuando Alan Turing hace unas contribuciones con un Paper \textit{Computing machinery and intelligence}, donde este señor ya estaba pensando en aprendizaje automático, algoritmos genéricos y aprendizaje por refuerzo.
            \\\\Seguido a este aporte de Alan Turing, se nos entrega el \textbf{Test de Turing}.
            \\\\El Test de Turing en resumidas cuentas, tiene a dos humanos, donde uno de ellos envía preguntas las cuales responde un humano y una máquina pero, el test dice que cuando el entrevistador humano no sabe quién contesto la pregunta, se puede decir que la inteligenia artifiial llego a un grado importante engañando al humano entrevistador. Hablamos de 1950.
            \\La IA como término nace en 1956 por \textit{John Macartey} llamando este enfoque como Inteligencia artificial "\textbf{siendo la ciencia de crear máquinas inteligentes, programas de computo inteligentes}".
            \\\\El uso del Test de Turing es una prueba para darse cuenta que tan avanzados están los algoritmos de IA después de un entrenamiento.
            \\\\En 1967 nace el Perceptrón Mark 1, este es básicamente un conjunto de redes neuronales, con entradas a realizar un proceso y dar una salida. El Perceptrón esta centrado a una nuestra naturaleza humanaes decir, el perceptrón tiene unas entradas, trabaja como unas neuronas se tratase y con Deep Learning. Entre más neuronas hay desventajas y ventajas como mayor precision más consumo de máquina.
            \\\\Debido a esto procesar media pagina vs 10 paginas, es una tarea que puede ser pesada en terminos de precisión. En Deep Learning tenemos muchas neuronas conectadas entre si para poder aprender mediante algoritmos.
            \\Hoy se habla del concepto Capas de Neuronas que se comunican y dan una respuesta 0 o 1.
            \\\\1980 Nace la Habitación China y es la antítesis del Test de Turing, ya que se refuta su idea de determinar el punto al que ha llegado una IA, porque dice que una máquina que pasa el Test de Turing no quiere decir que tenga inteligencia, sólo la simula. Por tanto, no entendió el contenido semántico la inteligencia artifical, no comprendió lo sintáctico.
            \begin{itemize}
              \item Eventos importantes
                \begin{enumerate}
                  \item IBM 1997 vence al campeón de ajedrez
                  \item IBM 2011 vence al campeón de jeopardy
                  \item MINWA 2015 super imagenes
                  \item 2017 nace la base de los generadores de texto, todos de sobre el principio de generacion de texto, que paso atras y que paso a futuro
              \end{enumerate}
            \end{itemize}
            Importante para estos generadores de texto del 2017, \textbf{Question Answer} competencias de IA a desarrollar problemas.
            \\Hablamosde ChatGPT y generadores de texto, construir sistemas que ayudan a alimentar los algoritmos mediante pruebas que ellas mismas como IA deben de resolver.
            \\\\El corpus es el base de conocimiento, y siguen aprendiendo.
            \\La programacion funcional es muy rapida y usa muchos parentesis ya que generan tiempos muy eficientes de computo.
            \\\\Inteligencia artificial, como la rama de la ciencia de la computación que se ocupa de la automatizacion dejando una entrada, proceos y salida automatizando la conducta inteligente Luger 2005.
            \\La inteligencia artificial tiene como objetivo las capacidades inteligentes, con influencia de la filosofia, matematica, psicologia, generar ideas enfoncadas en resolucion de problemas como areas especificas, Ejemplo ¿Quien mueve el automovil autonomamente?, para medir la IA y sus capacidades.

no es solo python,tiene muchas diciplinas involucradas

Enfoques, pero siguen siendo sistemas ia:
robots que piensan como humanos,
actuan como humanos
logica formal, sistemas expertos sistema que pueda tener un sensor enviar correo electronico, detectar enfermedades
Agentes, asistentes virtuales actuan racionalmente
            \\
%----------------------------------------------------------------------
    \part{PLN en Java}
    \part{Introducción a Python}
    \part{Bases PLN}
    \part{NLP - Pipeline}
    \part{Text Representation}
    \part{Text Clasification}
    \part{Information Extraction}
    \part{Spech Recognition and Synthesis}
    \part{Chatbots}
    \part{Search and Information Retrieval}
    \part{Topic Modelling}
    \part{Recommender Systems for Textual Data}
%======================================================================
    \part{Metodología de Calificación}
      \chapter{Cortes}
        \section{Porcentajes evaluativos}
          \paragraph{Cortes 1, 2, 3}\mbox{}\\
            \begin{itemize}
              \item Porcentajes
              \begin{enumerate}
                \item Talleres 20\%
                \item Proyecto 20\%
                \item Parcial-1 20\%
                \item Parcial-2 20\%
                \item Parcial-3 20\%
              \end{enumerate}
            \end{itemize}
%======================================================================
\end{document}

